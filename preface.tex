%%%%%%%%%%%%%%%%%%%%%%preface.tex%%%%%%%%%%%%%%%%%%%%%%%%%%%%%%%%%%%%%%%%%
% sample preface
%
% Use this file as a template for your own input.
%
%%%%%%%%%%%%%%%%%%%%%%%% Springer %%%%%%%%%%%%%%%%%%%%%%%%%%

\preface

This cheat sheet is a condensed version of machine learning manual, which contains many classical equations and diagrams on machine learning, and aims to help you quickly recall knowledge and ideas in machine learning.

This cheat sheet has two significant advantages:
\begin{enumerate}
\item Clearer symbols. Mathematical formulas use quite a lot of confusing symbols. For example, $X$ can be a set, a random variable, or a matrix. This is very confusing and makes it very difficult for readers to understand the meaning of math formulas. This cheat sheet tries to standardize the usage of symbols, and all symbols are clearly pre-defined, see section \S \ref{sec:Notation}.
\item Less thinking jumps. In many machine learning books, authors omit some intermediary steps of a mathematical proof process, which may save some space but causes difficulty for readers to understand this formula and readers get lost in the middle way of the derivation process. This cheat sheet tries to keep important intermediary steps as where as possible.
\end{enumerate}

