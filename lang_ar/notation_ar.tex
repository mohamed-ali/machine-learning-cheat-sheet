\Extrachap{مجموع الرموز \hfill \LR{Notation}}
\label{sec: \LR{Notation} \hfill مجموع الرموز}

\section*{تمهيد \hfill \LR{Introduction}}

%\LR{(data)}
%\LR{(models)}
%\LR{(algorithms)}
%\LR{(machine learning)}
%\LR{(statistics)}

من الصعوبة التّوصل إلي مجموعة وحيدة و ثابتة من الرموز لتغطية المجال الشاسع من البيانات \LR{(data)} و النماذج \LR{(models)} و الخوارزميات \LR{(algorithms)} التي نناقشها في هذا الكتيّب. علاوة على ذلك،  العلامات الرياضية المتفق عليها تختلف بين علم تعلّم الآلة \LR{(machine learning)} و علم الإحصاءات \LR{(statistics)}،  و بين الكتب والأوراق العلميّة المختلفة. مع ذلك، فقد حاولنا أن تكون الرموز المستعملة  متّسقة قدر الإمكان. فيما يلي نلخّص معظم الرموز المستخدمة, هذا لا ينفي أن بعض المقاطع الفرديّة في الكتيّب قد تعرض رموزا جديدة. إعلم أيضا أن بعض الرموز قد يكون لها معان مختلفة تبعا للسياق, رغم أننا سنحرص على تجنّب ذلك قدر الإمكان.

\section*{مجموع الرموز الرياضية \hfill \LR{General math notation}}

\begin{english}


\begin{longtable}{ll}
\hline\noalign{\smallskip}
\textbf{Symbol} & \textbf{Meaning} \\
\noalign{\smallskip}\hline\noalign{\smallskip}
$\lfloor x \rfloor$ & Floor of $x$, i.e., round down to nearest integer\\
$\lceil x \rceil$ & Ceiling of $x$, i.e., round up to nearest integer\\
$\vec{x} \otimes \vec{y}$ & Convolution of $\vec{x}$ and $\vec{y}$\\
$\vec{x} \odot \vec{y}$ & Hadamard (elementwise) product of $\vec{x}$ and $\vec{y}$\\
$a \wedge b$ & logical AND\\
$a \vee b$ & logical OR\\
$\neg a$ & logical NOT\\
$\mathbb{I}(x)$ & Indicator function, $\mathbb{I}(x)=1$ if x is true, else $\mathbb{I}(x)=0$\\
$\infty$ & Infinity\\
$\rightarrow$ & Tends towards, e.g., $n \rightarrow \infty$\\س
$\propto$ &Proportional to, so $y = ax$ can be written as $y \propto x$\\
$\abs{x}$ & Absolute value\\
$\abs{\mathcal{S}}$ & Size (cardinality) of a set\\
$n!$ & Factorial function\\
$\nabla$ & Vector of first derivatives\\
$\nabla^2$ & Hessian matrix of second derivatives\\
$\triangleq$ & Defined as\\
$O(\cdot)$ & Big-O: roughly means order of magnitude\\
$\mathbb{R}$ & The real numbers\\
$1:n$ & Range (Matlab convention): $1:n = {1, 2,...,n}$\\
$\approx$ & Approximately equal to\\
$\arg\max\limits_x f(x)$ & Argmax: the value $x$ that maximizes $f$\\
$B(a,b)$ & Beta function, $B(a,b)=\dfrac{\Gamma(a)\Gamma(b)}{\Gamma(a+b)}$\\
$B(\vec{\alpha})$ & Multivariate beta function, $\dfrac{\prod\limits_k \Gamma(\alpha_k)}{\Gamma(\sum\limits_k \alpha_k)}$\\
$\binom{n}{k}$ & $n$ choose $k$ , equal to $n!/(k!(n−k )!)$\\
$\delta(x)$ & Dirac delta function,$\delta(x)=\infty$ if $x=0$, else $\delta(x)=0$\\
$\exp(x)$ & Exponential function $e^x$\\
$\Gamma(x)$ & Gamma function, $\Gamma(x)=\int_0^\infty u^{x-1}e^{-u}\mathrm{d}u$\\
$\Psi(x)$ &  Digamma function,$Psi(x)=\dfrac{d}{dx}\log\Gamma(x)$\\
$\mathcal{X}$ & A set from which values are drawn (e.g.,$\mathcal{X}=\mathbb{R}^D$)\\
\noalign{\smallskip}\hline\noalign{\smallskip}
\end{longtable}
\end{english}

\section*{رموز علم الجبر الخطي \hfill \LR{Linear algebra notation}}

خلال هذا الكتاب, سنستخدم حروف boldface الصغيرة للدلالة على النّاقلات(vectors) مثل $\vec{x}$ و حروف boldface العلوية لدلالة على المصفوفات مثل  $\vec{X}$. نشير إلى إدخلات مصفوفة \LR{(matrix entries)} بأحرف كبيرة غير جريئة، مثل $X_{ij}$.

سنعتبر كل النّاقلات (vectors) ناقلات عمود \LR{(column vectors)}, مالم يذكر خلاف ذلك في السياق. نستخدم $(x_1,\cdots,x_D)$ للدّلالة على متجه عمود تم إنشاؤه بواسطة تكديس $D$ أعداد \LR{(stacking $D$ scalars)}. إذا كتبنا $\vec{X}=(\vec{x}_1,\cdots,\vec{x}_n)$, حيث الجانب الأيسر هو مصفوفة(matrix) , فإننا نقصد تكديس  $\vec{x}_i$ على طول الأعمدة لخلق مصفوفة(matrix). 
\begin{english}

\begin{longtable}{ll}
\hline\noalign{\smallskip}
\textbf{Symbol} & \textbf{Meaning} \\
\noalign{\smallskip}\hline\noalign{\smallskip}
$\vec{X} \succ 0$ & $\vec{X}$ is a positive definite matrix\\
$tr(\vec{X})$ & Trace of a matrix\\
$det(\vec{X})$ & Determinant of matrix $\vec{X}$\\
$\abs{\vec{X}}$ & Determinant of matrix $\vec{X}$\\
$\vec{X}^{-1}$ & Inverse of a matrix\\
$\vec{X}^{\dagger}$ & Pseudo-inverse of a matrix\\
$\vec{X}^T$ & Transpose of a matrix\\
$\vec{x}^T$ & Transpose of a vector\\
$\mathrm{diag}(x)$ & Diagonal matrix made from vector $\vec{x}$\\
$\mathrm{diag}(X)$ & Diagonal vector extracted from matrix $\vec{X}$\\
$\vec{I}$ or $\vec{I}_d$ & Identity matrix of size $d \times d$ (ones on diagonal, zeros of)\\
$\vec{1}$ or $\vec{1}_d$ & Vector of ones (of length $d$)\\
$\vec{0}$ or $\vec{0}_d$ & Vector of zeros (of length $d$)\\
$\abs{\abs{\vec{x}}}=\abs{\abs{\vec{x}}}_2$ & Euclidean or $\ell_2$ norm $\sqrt{\sum\limits_{j=1}^{d} x_j^2}$\\
$\abs{\abs{\vec{x}}}_1$ & $\ell_1$ norm $\sum\limits_{j=1}^{d} \abs{x_j}$\\
$\vec{X}_{:,j}$ & j'th column of matrix\\
$\vec{X}_{i,:}$ & transpose of $i$'th row of matrix (a column vector)\\
$\vec{X}_{i,j}$ & Element $(i,j)$ of matrix $\vec{X}$ \\
$\vec{x} \otimes \vec{y}$ & Tensor product of $\vec{x}$ and $\vec{y}$\\
\noalign{\smallskip}\hline\noalign{\smallskip}
\end{longtable}
\end{english}

\section*{رموز علم الإحتمال \hfill \LR{Probability notation}}

نرمز إلى الأعداد العشوائية و الثابتة  \LR{(random and fixed scalars)} بخط صغير \LR{(lower case)}, و الناقلات العشوائية و الثابتة \LR{(random and fixed vectors)} بالحروف الصغيرة الجريئة\LR{(bold lower case)} و المصفوفات العشوائية و الثابتة \LR{(random and fixed matrices)} بالحروف الجريئة العلوية \LR{(bold upper case)}. أحيانا نستخدم الحروف العلوية غير الجريئة \LR{(non-bold upper case)} لدلالة على المتغيرات العددية العشوائية \LR{(scalar random variables)}. نستخدم, أيضا, $p()$ لكل من المتغيرات العشوائية المنفصلة و المستمرة \LR{(discrete and continuous random variables)}. 

\begin{english}

\begin{longtable}{ll}
\hline\noalign{\smallskip}
\textbf{Symbol} & \textbf{Meaning} \\
\noalign{\smallskip}\hline\noalign{\smallskip}
$X,Y$ & Random variable\\
$P()$ & Probability of a random event\\
$F()$ & Cumulative distribution function(CDF), also called distribution function\\
$p(x)$ & Probability mass function(PMF)\\
$f(x)$ & probability density function(PDF) \\
$F(x,y)$ & Joint CDF\\
$p(x,y)$ & Joint PMF \\
$f(x,y)$ & Joint PDF\\
$p(X|Y)$ & Conditional PMF, also called conditional probability\\
$f_{X|Y}(x|y)$ & Conditional PDF\\
$X \perp Y$ & X is independent of Y\\
$X \not\perp Y$ & X is not independent of Y\\
$X \perp Y | Z $ & X is conditionally independent of Y given Z\\
$X \not\perp Y | Z $ & X is not conditionally independent of Y given Z\\
$X \sim p$ & X is distributed according to distribution $p$\\
$\vec{\alpha}$ & Parameters of a Beta or Dirichlet distribution\\
$\mathrm{cov}[X]$ & Covariance of X\\
$\mathbb{E}[X]$ & Expected value of X\\
$\mathbb{E}_q[X]$ & Expected value of X wrt distribution $q$\\
$\mathbb{H}(X)$ or $\mathbb{H}(p)$ & Entropy of distribution $p(X)$\\
$\mathbb{I}(X;Y)$ & Mutual information between X and Y\\
$\mathbb{KL}(p||q)$ & KL divergence from distribution $p$ to $q$\\
$\ell(\vec{\theta})$ & Log-likelihood function\\
$L(\theta,a)$ & Loss function for taking action $a$ when true state of nature is $\theta$\\
$\lambda$ & Precision (inverse variance) $\lambda=1/\sigma^2$\\
$\Lambda$ & Precision matrix $\Lambda=\Sigma^{-1}$\\
mode$[\vec X]$ & Most probable value of $\vec X$\\
$\mu$ & Mean of a scalar distribution\\
$\vec{\mu}$ & Mean of a multivariate distribution\\
$\Phi$ & cdf of standard normal\\
$\phi$ & pdf of standard normal\\
$\vec{\pi}$ & multinomial parameter vector, Stationary distribution of Markov chain\\
$\rho$ & Correlation coefficient \\
sigm($x$) & Sigmoid (logistic) function,$\dfrac{1}{1+e^{-x}}$\\
$\sigma^2$ & Variance\\
$\Sigma$ & Covariance matrix\\
var[$x$] & Variance of $x$\\
$\nu$ & Degrees of freedom parameter\\
Z & Normalization constant of a probability distribution\\
\noalign{\smallskip}\hline\noalign{\smallskip}
\end{longtable}
\end{english}

\section*{رموز علم تعلّم الآلة و الإحصاءات \hfill \LR{Machine learning/statistics notation}}

\begin{english}
In general, we use upper case letters to denote constants, such as $C, K, M, N, T$, etc. We use lower case letters as dummy indexes of the appropriate range, such as $c=1:C$ to index classes, $i=1:M$ to index data cases, $j=1:N$ to index input features, $k=1:K$ to index states or clusters, $t=1:T$ to index time, etc.

We use $x$ to represent an observed data vector. In a supervised problem, we use $y$ or $\vec{y}$ to represent the desired output label. We use $\vec{z}$ to represent a hidden variable. Sometimes we also use $q$ to represent a hidden discrete variable.

\begin{longtable}{ll}
\hline\noalign{\smallskip}
\textbf{Symbol} & \textbf{Meaning} \\
\noalign{\smallskip}\hline\noalign{\smallskip}
$C$ & Number of classes\\
$D$ & Dimensionality of data vector (number of features)\\
$N$ & Number of data cases\\
$N_c$ & Number of examples of class $c$,$N_c=\sum_{i=1}^{N}\mathbb{I}(y_i=c)$\\
$R$ & Number of outputs (response variables)\\
$\mathcal{D}$ & Training data $\mathcal{D}=\left\{(\vec{x}_i,y_i) | i=1:N\right\}$\\
$\mathcal{D}_{test}$ & Test data\\
$\mathcal{X}$ & Input space\\
$\mathcal{Y}$ & Output space\\
$K$ & Number of states or dimensions of a variable (often latent)\\
$k(x,y)$ & Kernel function\\
$\vec{K}$ & Kernel matrix\\
$\mathcal{H}$ & Hypothesis space\\
$L$ & Loss function \\
$J(\vec{\theta})$ & Cost function\\
$f(\vec{x})$ & Decision function\\
$P(y|\vec{x})$ & Conditional probability\\
$\lambda$ & Strength of $\ell_2$ or $\ell_1 regularizer$\\
$\phi(x)$ & Basis function expansion of feature vector $\vec{x}$\\
$\Phi$ & Basis function expansion of design matrix $\vec{X}$\\
$q()$ & Approximate or proposal distribution\\
$Q(\vec{\theta},\vec{\theta}_{old})$ & Auxiliary function in EM\\
$T$ & Length of a sequence\\
$T(\mathcal{D})$ & Test statistic for data\\
$\vec{T}$ & Transition matrix of Markov chain\\
$\vec{\theta}$ & Parameter vector\\
$\vec{\theta}^{(s)}$ & $s$'th sample of parameter vector\\
$\hat{\vec{\theta}}$ & Estimate (usually MLE or MAP) of $\vec{\theta}$\\
$\hat{\vec{\theta}}_{MLE}$ & Maximum likelihood estimate of $\vec{\theta}$\\
$\hat{\vec{\theta}}_{MAP}$ & MAP estimate of $\vec{\theta}$\\
$\bar{\vec{\theta}}$ & Estimate (usually posterior mean) of  $\vec{\theta}$\\
$\vec{w}$ & Vector of regression weights (called $\vec{\beta}$ in statistics)\\
b & intercept (called $\varepsilon$ in statistics)\\
$\vec{W}$ & Matrix of regression weights\\
$x_{ij}$ & Component (i.e., feature) $j$ of data case $i$ ,for $i=1:N ,j=1:D$\\
$\vec{x}_i$ & Training case, $i=1:N$\\
$\vec{X}$ & Design matrix of size $N \times D$\\
$\bar{\vec{x}}$ & Empirical mean $\bar{\vec{x}}=\dfrac{1}{N}\sum_{i=1}^{N} \vec{x}_i$\\
$\tilde{\vec{x}}$ & Future test case\\
$\vec{x}_*$ & Feature test case\\
$\vec{y}$ & Vector of all training labels $\vec{y} =(y_1,...,y_N)$\\
$z_{ij}$ & Latent component $j$ for case $i$\\
\noalign{\smallskip}\hline\noalign{\smallskip}
\end{longtable}

\end{english}

%\twocolumn
